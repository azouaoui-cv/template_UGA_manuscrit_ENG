%%%%%%%%%%%%%%%%%%%%%%%%%%%%%%%%%%%%%%%%%%%%%%%%%%%%%%%%%%%%%%%%%%%%%%%%%%

%%%%%                           General Conclusion                     %%%%%%
%%%%%%%%%%%%%%%%%%%%%%%%%%%%%%%%%%%%%%%%%%%%%%%%%%%%%%%%%%%%%%%%%%%%%%%%%%

\chapter{Conclusion and perspectives}
\label{ch:ccl}
%\end{flushright}
%\vspace{2cm}

\lhead[\fancyplain{}{Conclusion and perspectives}]
      {\fancyplain{}{}}
\chead[\fancyplain{}{}]
      {\fancyplain{}{}}
\rhead[\fancyplain{}{}] 
      {\fancyplain{}{Conclusion and perspectives}}
\lfoot[\fancyplain{}{}]
      {\fancyplain{}{}}
\cfoot[\fancyplain{}{\thepage}]
      {\fancyplain{}{\thepage}}
\rfoot[\fancyplain{}{}]
     {\fancyplain{}{\scriptsize}}

%%%%%%%%%%%%%%%%%%%%%%%%%%%%%%%%%%%%%%%%%%%%%%%%%%%%%%%%%%%%%%%%%%%%%%%%%%
%%%%%                      Start part here                          %%%%%%
%%%%%%%%%%%%%%%%%%%%%%%%%%%%%%%%%%%%%%%%%%%%%%%%%%%%%%%%%%%%%%%%%%%%%%%%%%

In this thesis, we have developed efficient and interpretable algorithms aimed at addressing essential challenges in hyperspectral data analysis, specifically focusing on image restoration and spectral unmixing.

First, we introduced a novel sparse coding-based unfolding algorithm designed for the representation of hyperspectral images.

Furthermore, we leveraged an established model-based framework known as archetypal analysis to tackle various unmixing scenarios, offering valuable insights and solutions for spectral unmixing tasks.

To facilitate the practical application of these methods, we have also developed a Python-based toolbox tailored for hyperspectral unmixing.
This toolbox empowers practitioners to readily evaluate and compare multiple unmixing techniques on their own datasets, enhancing accessibility and usability in the hyperspectral research community.

\section{Contributions summary}

Here we summarize the contributions presented in this thesis.

\paragraph{Unfolding sparse coding to restore hyperspectral images}

Our first contribution investigate data-efficient and interpretable models for hyperspectral image restoration, by encoding prior knowledge used in the dictionary learning literature into an end-to-end trainable network architecture.
In Chapter \ref{ch:T3SC}, we introduce a novel two-layered model-based deep learning method called T3SC. 
The primary objective of T3SC is to provide effective denoising of hyperspectral images while maintaining interpretability.
The efficiency of our approach proved to be key to handle the limited availability of training data for hyperspectral denoising.
In addition, our principled architecture, namely the sensor specific first spectral layer followed by a sensor agnostic spectral-spatial layer, is tailored to the specificity of hyperspectral images that are typically acquired by various sensors with different characteristics.

\paragraph{Modeling spectral unmixing using archetypal analysis}

Our second and third contributions draw significant inspiration from the interpretability offered by the archetypal analysis framework.
In Chapter \ref{ch:EDAA} we demonstrate the viability of archetypal analysis for blind hyperspectral unmixing within the linear mixing model.
To efficiently tackle this challenge, we introduce a novel optimization scheme known as entropic gradient descent, specifically tailored for solving the archetypal analysis formulation.
Notably, our approach is highly compatible with GPU computation, and its computational efficiency enables us to develop a robust model selection procedure that mitigates the challenges posed by hyperparameter choices.
In Chapter \ref{ch:SUnAA} we extend our approach to semi-supervised unmixing, emphasizing its advantages in addressing discrepancies between endmembers extracted from a library and those encountered in the actual scene.
To facilitate the optimization process, we employ an active-set algorithm that leverages the inherent sparsity of solutions within the quadratic program, while enforcing simplicial constraints.


\paragraph{Enabling simple benchmarks for hyperspectral unmixing}
Finally, in Chapter \ref{ch:HySUPP}, we draw a critical comparison of different unmixing techniques, classified into supervised, semi-supervised and blind unmixing categories.
To enhance accessibility and usability, we introduce an open-source Python toolbox that offers users a wide array of unmixing techniques suitable for their specific datasets.
This toolbox facilitates benchmarking across a range of popular approaches using both simulated and real datasets, enabling a comprehensive assessment of their respective merits and limitations.


\section{Future research and perspectives}

Based on the contributions of this thesis, several questions and research directions arise and would be interesting to investigate in the future.

\paragraph{Remaining challenges in spectral unmixing}

Despite significant advancements, spectral unmixing remains one of the most challenging tasks in hyperspectral analysis.
Below, we briefly outline some of the primary challenges encountered in spectral unmixing:
\begin{itemize}
    \item Linear models, while generally versatile, may experience a significant drop in performance when applied to different datasets.
    \item The selection of appropriate parameters significantly impacts the effectiveness of unmixing methods, and optimizing them for real-world datasets is particularly challenging.
    \item Linear unmixing methods often exhibit degraded performance as the number of endmembers increases, potentially failing on datasets with numerous endmembers.
    \item Spectral variability poses a major challenge, as it can significantly reduce the performance of linear unmixing techniques.
    \item The absence of real datasets with ground truth is a significant impediment.
    \item Multi-temporal and multi-source spectral unmixing tasks present additional complexities.
    \item Given the large volume of HS data, scalable unmixing approaches are crucial for global monitoring.
\end{itemize}

These challenges highlight the ongoing efforts and research needed to enhance the robustness and adaptability of spectral unmixing methods in various applications.