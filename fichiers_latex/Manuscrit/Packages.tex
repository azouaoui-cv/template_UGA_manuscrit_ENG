% %%%%%%%%%%%%%%%%%%%%%%%%%%%%%%%%%%%%%%%%%%%%%%%%%%%%%%%%%%%%%%%%%%%%%%%%%%
% %                            PAQUETS USUELS                              %
% %                                                   %
% %%%%%%%%%%%%%%%%%%%%%%%%%%%%%%%%%%%%%%%%%%%%%%%%%%%%%%%%%%%%%%%%%%%%%%%%%%
%% Package for line new first page
\usepackage{xparse}
\usepackage{mdframed}
\newmdenv[
    topline=false,
    bottomline=false,
    rightline=false,
    linewidth=2pt,
    linecolor=orange,
    innerleftmargin=5pt,
    leftmargin=90pt,
    rightmargin=0pt,
    innerrightmargin=0pt,
]{textbox}


\usepackage[T1]{fontenc}
\usepackage[utf8]{inputenc}
\usepackage[french,english]{babel}
%\usepackage{babel}  % francais = french babel
\usepackage{meta-donnees} % page de garde
\usepackage{meta-donnees2} %page de 4ième de couv
\usepackage{amssymb}
\usepackage{verbatim}
\usepackage{array}	
\usepackage{color}
\usepackage{cite}
\usepackage{xfrac}
\usepackage{mathptmx}
\usepackage{enumitem}
\usepackage{esvect}
\usepackage[squaren,Gray]{SIunits}
\usepackage{sistyle}
\usepackage{eurosym}  %pour obtenir le symbole Euro
%\usepackage{gensymb} %pour obtenir le symbole \degree
\usepackage{calligra}
\usepackage{ragged2e} % for justiier text
\usepackage{tikz}
\usetikzlibrary{positioning,backgrounds,fadings,shadows.blur,shadows}
\usetikzlibrary{fit,shapes.misc}
\usepackage{xcolor}
\usepackage{animate}

\usepackage{blindtext}

%%%%%%%%%%%%%%%%%%%%%%%%%%%%%%%%%%%%%%%%%%%%%%%%%%%%%%%%%%%%%%%%%%%%%%%%%%
%%%%%           Packages pour les entetes et pied de page           %%%%%%
%%%%%%%%%%%%%%%%%%%%%%%%%%%%%%%%%%%%%%%%%%%%%%%%%%%%%%%%%%%%%%%%%%%%%%%%%%

%\usepackage{picins}
\usepackage{fancyhdr}
%\usepackage{psboxit}  %%% A Inserer avant babel !!!! 
\usepackage{pifont}


%%%%%%%%%%%%%%%%%%%%%%%%%%%%%%%%%%%%%%%%%%%%%%%%%%%%%%%%%%%%%%%%%%%%%%%%%%
%%%%%                   Packages et couleurs perso                  %%%%%%
%%%%%%%%%%%%%%%%%%%%%%%%%%%%%%%%%%%%%%%%%%%%%%%%%%%%%%%%%%%%%%%%%%%%%%%%%%

\usepackage{xcolor}  %%% Incompatibilité avec \usepackage{colortbl} ??????
\definecolor{BleuCyan}{RGB}{0,190,190}  %%% Définition d'une couleur personnelle
\definecolor{RoseRose}{RGB}{238,44,44}
\definecolor{VertVert}{RGB}{10,255,118}
\definecolor{Anthracite}{RGB}{91,124,151}
\definecolor{GrisPasTropClair}{RGB}{83,135,135}
\definecolor{BleuPetrole}{RGB}{0, 0 ,205}
\definecolor{BleuClair}{RGB}{234, 255 ,255}
\definecolor{Bleu1}{RGB}{26, 64 ,145}
\definecolor{Rouge1}{RGB}{215, 19 ,24}
\definecolor{myblue}{rgb}{.8, .8, 1}
\definecolor{violet1}{rgb}{0.78,0.53,0.97}
\definecolor{myblue2}{rgb}{0,0.41,0.54} % dark blue
\definecolor{myred}{RGB}{192,0,0} % dark red
\definecolor{mygreen2}{RGB}{0,120,0} % dark green
\definecolor{myblue3}{HTML}{1a4091} % dark blue 	
\definecolor{couleur_marie}{RGB}{0,102,204} % dark green





%\newcommand*\maboite[1]{%
%\fcolorbox{GrisPasTropClair}{BleuClair}{\hspace{1em}#1\hspace{1em}}}
%\newcommand{\parttoccolor}{blue}
%\newcommand{\chaptertoccolor}{red}
%\newcommand{\sectiontoccolor}{green!70!black}



%%%%%%%%%%%%%%%%%%%%%%%%%%%%%%%%%%%%%%%%%%%%%%%%%%%%%%%%%%%%%%%%%%%%%%%%%%
%%%%%            Packages pour les captions optimisés               %%%%%%
%%%%%%%%%%%%%%%%%%%%%%%%%%%%%%%%%%%%%%%%%%%%%%%%%%%%%%%%%%%%%%%%%%%%%%%%%%
%\usepackage[font=small,font={it}]{caption} % \usepackage[small,hang]{caption2} apparement cpation2 est obsolette
\usepackage[labelfont=bf,font={normal}]{caption}
%\captionsetup[table]{position=bottom}
%\renewcommand{\captionfont}{\it \small}
%\renewcommand{\captionlabelfont}{\it \bf \small}
% \renewcommand{\captionlabeldelim}{ :}  % Ne marche qu'avec caption2



%%%%%%%%%%%%%%%%%%%%%%%%%%%%%%%%%%%%%%%%%%%%%%%%%%%%%%%%%%%%%%%%%%%%%%%%%%
%%%%%           Packages pour les titres de chapitres               %%%%%%
%%%%%%%%%%%%%%%%%%%%%%%%%%%%%%%%%%%%%%%%%%%%%%%%%%%%%%%%%%%%%%%%%%%%%%%%%%
%\usepackage[Bjornstrup]{fncychap}
\usepackage[avantgarde]{quotchap}

\usepackage{lettrine}    %Lettrine exemple: \lettrine[lines=2]{L}{orem ipsum}
%\usepackage[francais]{minitoc}		% Pour ajouter une table des matières à chaque chapitre
\usepackage{minitoc}		% Pour ajouter une table des matières à chaque chapitre

\setcounter{minitocdepth}{2}

%%%%%%%%%%%%%%%%%%%%%%%%%%%%%%%%%%%%%%%%%%%%%%%%%%%%%%%%%%%%%%%%%%%%%%%%%%
%%%%%                Packages pour les figures                      %%%%%%
%%%%%%%%%%%%%%%%%%%%%%%%%%%%%%%%%%%%%%%%%%%%%%%%%%%%%%%%%%%%%%%%%%%%%%%%%%
\usepackage{epsfig}
\usepackage{wrapfig}  %%% Inserer du texte a droite ou à gauche de l'image
%\usepackage{picins}  %%% Inserer du texte a droite ou à gauche de l'image
\usepackage{float}
\usepackage{graphicx}
%\usepackage{subfigure}
\usepackage{subcaption}
%\usepackage{placeins}
\usepackage{hhline}
\usepackage{colortbl}
%\usepackage{mwe}
\usepackage{float} % Permite um melhor posicionamento de gráficos
\usepackage{svg}
%%%%%%%%%%%%%%%%%%%%%%%%%%%%%%%%%%%%%%%%%%%%%%%%%%%%%%%%%%%%%%%%%%%%%%%%%%
%%%%%                  Packages pour les tableaux                   %%%%%%
%%%%%%%%%%%%%%%%%%%%%%%%%%%%%%%%%%%%%%%%%%%%%%%%%%%%%%%%%%%%%%%%%%%%%%%%%%
\usepackage{array}
\usepackage{textcomp}
\usepackage{booktabs}
\usepackage{colortbl}  %%% Couleurs de cellule de tableau

\usepackage{lscape}    %%% Rotation des tableaux
%\usepackage{multirow}
\usepackage{tabularx}
%\usepackage{slashbox}
\usepackage{pifont}
%\usepackage[table]{xcolor}
\usepackage{tablefootnote}
\usepackage{threeparttable}
\usepackage{colortbl}
\usepackage{vcell}
\arrayrulecolor{myblue3}
\let\oldtabular=\tabular 
\def\tabular{\small\oldtabular}
\usepackage{longtable}
\usepackage[longtable]{multirow}
%\usepackage{longtable}
\usepackage{adjustbox}
 \usepackage{makecell}
 
%%%%%%%%%%%%%%%%%%%%%%%%%%%%%%%%%%%%%%%%%%%%%%%%%%%%%%%%%%%%%%%%%%%%%%%%%%
%%%%%                  Packages pour les maths                  %%%%%%
%%%%%%%%%%%%%%%%%%%%%%%%%%%%%%%%%%%%%%%%%%%%%%%%%%%%%%%%%%%%%%%%%%%%%%%%%%
\usepackage{amsmath}
\usepackage{bm}
%\usepackage{stix}
%\usepackage{empheq} % Pour encadrer les equations
%pour utiliser plusieurs fichiers bibteX
%\usepackage{biblist}
\newcommand{\bDiamond}{\mathbin{\Diamond}}
\newcommand{\bLozenge}{\mathbin{\blacklozenge}}


%\usepackage{fourier}
\usepackage{SIunits}



%%%%%%%%%%%%%%%%%%%%%%%%%%%%%%%%%%%%%%%%%%%%%%%%%%%%%%%%%%%%%%%%%%%%%%%%%%
%%%%%              Packages pour les liens hypertexte               %%%%%%
%%%%%%%%%%%%%%%%%%%%%%%%%%%%%%%%%%%%%%%%%%%%%%%%%%%%%%%%%%%%%%%%%%%%%%%%%%
\usepackage{hyperref} %%%  Packages pour les liens hypertexte a mettre après tous les autres packages
 \hypersetup{
 colorlinks=true,
 %linkcolor=BleuCyan,
 linkcolor=couleur_marie,
 citecolor=BleuPetrole,
 filecolor=VertVert
 }
 
%  hyperindex=true,

%%%%%%%%%%%%%%%%%%%%%%% others


%%%% test commands

% \usepackage{showframe}
% \renewcommand\ShowFrameLinethickness{0.15pt}
% \renewcommand*\ShowFrameColor{\color{red}}
