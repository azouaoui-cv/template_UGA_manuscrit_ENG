%%%%%%%%%%%%%%%%%%%%%%%%%%%%%%%%%%%%%%%%%%%%%%%%%%%%%%%%%%%%%%%%%%%%%%%%%%
%%%%%                         CHAPITRE 1                            %%%%%%
%%%%%%%%%%%%%%%%%%%%%%%%%%%%%%%%%%%%%%%%%%%%%%%%%%%%%%%%%%%%%%%%%%%%%%%%%%

\lhead[\fancyplain{}{\leftmark}]%Pour les pages paires \bfseries
      {\fancyplain{}{}} %Pour les pages impaires
\chead[\fancyplain{}{}]%
      {\fancyplain{}{}}
\rhead[\fancyplain{}{}]%Pour les pages paires 
      {\fancyplain{}{\rightmark}}%Pour les pages impaires \bfseries
\lfoot[\fancyplain{}{}]%
      {\fancyplain{}{}}
\cfoot[\fancyplain{}{\thepage}]%\bfseries
      {\fancyplain{}{\thepage}} %\bfseries
\rfoot[\fancyplain{}{}]%
     {\fancyplain{}{\scriptsize}}


%%%%%%%%%%%%%%%%%%%%%%%%%%%%%%%%%%%%%%%%%%%%%%%%%%%%%%%%%%%%%%%%%%%%%%%%%%
%%%%%                      Start part here                          %%%%%%
%%%%%%%%%%%%%%%%%%%%%%%%%%%%%%%%%%%%%%%%%%%%%%%%%%%%%%%%%%%%%%%%%%%%%%%%%%

\chapter{title 1}
\label{ch:1}

%==============================================================================	Summary of the chapter

\begin{center}
\rule{0.7\linewidth}{.5pt}
\begin{minipage}{0.7\linewidth}
\smallskip

\textit{The resume of the chapter include here.
}

%\smallskip
\end{minipage}
\smallskip
\rule{0.7\linewidth}{.5pt}
\end{center}

\minitoc
\newpage


\section{Examples}
\subsection{References}

Example of citation \cite{Collomb2017}, or like  \cite{Collomb2018b},or like that \cite{Collomb2018a,Collomb2017a,Collomb2018}. \\


\FloatBarrier
\subsection{Figures}

Figure~\ref{figure_simple} is an example of the integration of one figure simple. \\

\begin{figure}[ht]
	\centering
	\def\svgwidth{1\columnwidth}
	\fontsize{10pt}{10pt}\selectfont\input{../Chap1/Figure/Figure_1.pdf_tex}
	\caption{Example of figure}
	\label{figure_simple}
\end{figure}


Figure~\ref{figure_multiple} is one example of the integration of multiple figures.\\ Also is possible to make the reference of the figure like Figure~\ref{figure_subplot} or like Figure~\ref{figure_polaire}.

\begin{figure}[ht]
	\centering
	\begin{subfigure}[b]{0.8\textwidth}
		\centering
		\def\svgwidth{\columnwidth}
		\fontsize{10pt}{10pt}\selectfont\input{../Chap1/Figure/Figure_2.pdf_tex}
		\caption{Example subplot} 
		\label{figure_subplot}
	\end{subfigure}
	\qquad
	\begin{subfigure}[b]{0.7\textwidth}
		\centering
		\def\svgwidth{\columnwidth}
		\fontsize{10pt}{10pt}\selectfont\input{../Chap1/Figure/Figure_3.pdf_tex}
		\caption{Example diagram polar } 
		\label{figure_polaire}
	\end{subfigure}
	\caption{Example of multiples Figures } 
	\label{figure_multiple}
\end{figure}



\FloatBarrier
\subsection{Tables}

Generate a table \href{http://www.tablesgenerator.com/latex_tables}{here}. \\

One example of the table generated by the website is presented here:

Table~\ref{tableau_exemple}.

\begin{table}[ht]
\centering
\begin{tabular}{c|c|c|c|}
\cline{2-4}
                               & \textbf{A}                 & \textbf{B} & \textbf{C} \\ \hline
\multicolumn{1}{|c|}{$\alpha$} & \multicolumn{3}{c|}{\textit{fusion}}                 \\ \hline
\multicolumn{1}{|c|}{$\beta$}  & \multirow{2}{*}{\textit{}} & \textit{1} & \textit{2} \\ \cline{1-1} \cline{3-4} 
\multicolumn{1}{|c|}{$\Delta$} &                            & \textit{3} & \textit{4} \\ \hline
\end{tabular}
\caption{Example of table}
\label{tableau_exemple}
\end{table}


\FloatBarrier
\section{Section 2}
\subsection{Sub Section 1}
\blindtext

\subsection{Sub Section 2}
\blindtext